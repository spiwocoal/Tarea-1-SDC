\section{Pregunta \texttt{i)}}\label{pregunta-i}




\begin{equation}
    \frac{ \nara{k_c k_a} \left( 0.009 z^2 + 0.0058 z - 0.0038 \right)}
    { z^5 -1.3895z^4 + 0.8998z^3 + \left( 0.009\nara{k_c k_a k_{st}}  -1.0775 \right) z^2 
    + \left(0.0058 \nara{k_c k_a k_{st}}   + 0.5673 \right) z 
    -0.0038 \nara{k_c k_a k_{st}} }
\end{equation}


Ahora bien para el calculo del $\nara{kc}$ en este apartado se procedera al igual que en el apartado \hyperref[pregunta-h]{\texttt{h)}}.

Comenzado para este caso, se tiene $G(z) = \frac{k_{c}}{z-1}\nara{ k_{a}} z^{-1} \cdot \frac{\rojo{\psi}(z)}{\verd{v_i}(z)}$,
y $r(z) = \nara{k_{st}}$.  Por lo tanto nuestra Funcion de transferencia de lazo cerrado sera,
\begin{equation}
    \frac{G(z)}{1 + G(z)r(z)} = \dfrac{\frac{k_{c}}{z-1}\nara{k_a} z^{-1} \cdot\frac{0.008997z^2 + 0.005767 z - 0.003827}
    {z^3 - 0.3895 z^2 + 0.5102 z -0.5673}}
    {1+\frac{k_{c}}{z-1}\nara{k_a} z^{-1} \frac{0.008997z^2 + 0.005767 z - 0.003827}
    {z^3 - 0.3895 z^2 + 0.5102 z -0.5673}\nara{k_{st}}}
\end{equation}

Reduciendo y simplificando se obtiene:
\begin{equation}\label{eq:fdt-LCdi}
    \frac{ \nara{k_c k_a} \left( 0.009 z^2 + 0.0058 z - 0.0038 \right)}
    { z^5 -1.3895z^4 + 0.8998z^3 + \left( 0.009\nara{k_c k_a k_{st}}  -1.0775 \right) z^2 
    + \left(0.0058 \nara{k_c k_a k_{st}}   + 0.5673 \right) z 
    -0.0038 \nara{k_c k_a k_{st}} }
\end{equation}


Nuevamente nos enfocamos entonces en el denominador que es donde se ecuentran lo polos y
utilizamos el criterio de Routh-Hurwitz para analizar la estabilidad. 
Trabajaremos entonces con el denominador de \eqref{eq:fdt-LCdi}, donde $\nara{a_4},\nara{a_3},\nara{a_2},\nara{a_1},\nara{a_0}$
representan los coeficientes del polinomio:
\begin{align}
    P(z) = z^5+ \nara{a_4}z^4 +\nara{a_3}z^3 + \nara{a_2} z^2 + \nara{a_1} z + \nara{a_0} &
\end{align}

Definimos la transformación bilineal, en la cual se reemplaza $z$, tal que 
\begin{equation}
    z = \frac{w+1}{1-w}
\end{equation}

Ahora sustituimos $z$ en el polinomio característico y simplificamos:
\begin{equation}
    P(z) = (\frac{w+1}{1-w})^5 +\nara{a_4}\frac{w+1}{1-w})^4 + \nara{a_3}(\frac{w+1}{1-w})^3 + \nara{a_2} (\frac{w+1}{1-w})^2 + \nara{a_1} (\frac{w+1}{1-w}) + \nara{a_0}
\end{equation}

Luego, se multiplica por $(1-w)^5$:
\begin{equation}
    P(z) =  ({w+1})^5 +\nara{a_4}({w+1})^4(1-w)+ \nara{a_3}({w+1})^3(1-w)^2 + \nara{a_2} ({w+1})^2 (1-w)^3+ \nara{a_1} ({w+1})(1-w)^4+ \nara{a_0}(1-w)^5
\end{equation}



Resolvemos los productos, distribuimos y reordenamos obteniendo:
\begin{align}
  P(z) &= (\nara{a_1} - \nara{a_0} - \nara{a_2} + \nara{a_3} - \nara{a_4} + 1) w^5 + (5\nara{a_0} - 3\nara{a_1} + \nara{a_2} + \nara{a_3} - 3\nara{a_4} + 5) w^4 \\
  &+ (2\nara{a_1} - 10\nara{a_0} + 2\nara{a_2} - 2\nara{a_3} - 2\nara{a_4} + 10) w^3 + (10\nara{a_0} + 2\nara{a_1} - 2\nara{a_2} - 2\nara{a_3} + 2\nara{a_4} + 10) w^2 \\
  &+ (\nara{a_3} - 3\nara{a_1} - \nara{a_2} - 5\nara{a_0} + 3\nara{a_4} + 5) w + (\nara{a_0} + \nara{a_1} + \nara{a_2} + \nara{a_3} + \nara{a_4} + 1)
\end{align}

para simplificar la tabla haremos los siguientes cambios de variables :
\begin{align}
    A_5&= (\nara{a_1} - \nara{a_0} - \nara{a_2} + \nara{a_3} - \nara{a_4} + 1)\\
    A_4 &=(5\nara{a_0} - 3\nara{a_1} + \nara{a_2} + \nara{a_3} - 3\nara{a_4} + 5) \\
    A_3 &= (2\nara{a_1} - 10\nara{a_0} + 2\nara{a_2} - 2\nara{a_3} - 2\nara{a_4} + 10)\\
    A_2 &=(10\nara{a_0} + 2\nara{a_1} - 2\nara{a_2} - 2\nara{a_3} + 2\nara{a_4} + 10)\\
    A_1 &=(\nara{a_3} - 3\nara{a_1} - \nara{a_2} - 5\nara{a_0} + 3\nara{a_4} + 5)\\
    A_0 &= (\nara{a_0} + \nara{a_1} + \nara{a_2} + \nara{a_3} + \nara{a_4} + 1)
\end{align}

Construimos entonces la tabla para el criterio de Routh-Hurwitz, y observamos
que no hay ceros presentes en la columan pivote:
\begin{equation}
  \begin{array}{c|ccc}
    w^5 & A_5 & A_3& A_1 \\
    w^4 & A_4 & A_2 & A_0 \\
    w^3 & \nara{b_3}  &  \nara{b_2} &0\\
    w^2 & \nara{b_1}  & A_0\\
    w^1 & \nara{b_0}\\
    w^0 & A_0
  \end{array}
\end{equation}

Donde sabemos que:
\begin{align}
    \nara{b_3} &= \frac{A_4 A_3 - A_5 A_2}{A_4},\\ 
    \nara{b_2} &= \frac{A_4 A_1 - A_5 A_0}{A_4},\\ 
    \nara{b_1} &= \frac{\nara{b_3} A_2 - A_4 \nara{b_2}}{\nara{b_3}},\\ 
    \nara{b_0} &= \frac{\nara{b_1} \nara{b_2} - \nara{b_3} A_0}{\nara{b_1}}.
  \end{align}

Para tener estabilidad en el sistema, los coeficientes de la primera columna
deben ser todos positivos. Entonces, la condición $\nara{b_3} \land \nara{b_2} \land\nara{b_1}\land\nara{b_0}>0$ garantiza
que el sistema vaya a ser estable.

Donde con \textit{MATLAB} obtenemos los valores de $\nara{k_c}$ que cumplen la condición por separado siendo los siguientes:

\vspace*{0.2cm}

    \begin{multicols}{2}
        Para $\nara{b_1}$: 
        \begin{align}
            \nara{k_{c0}} & = -0.0162 \\
            \nara{k_{c1}} & = 0.0128
        \end{align}

        \columnbreak
        
        Para $\nara{b_0}$:
        \begin{align}
            \nara{k_{c2}} & = 0.0069 \\
            \nara{k_{c3}} & = -0.0163 \\
            \nara{k_{c4}} & = -0.0163
        \end{align}
    \end{multicols}

Probando los valores donde \( k_c > 0 \) en \textit{MATLAB}, vemos cuál cumple la condición del criterio de Routh-Hurwitz, que consiste en que la columna pivote sea completamente positiva.

\begin{equation}
    \nara{k_{c}}  < 0.0069
\end{equation}

Ahora bien para que el sistema sea estable entonces se debe cumplir lo anterior. Pero 
como solo nos interesa que sea marginalmente estable, entonces el $\nara{k_{c}}$
que hace que el sistema lo sea es:
\begin{equation}
    \boxed{\nara{k_c} = 0.0069}
\end{equation}