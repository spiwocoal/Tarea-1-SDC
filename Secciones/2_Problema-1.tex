\section{Pregunta \texttt{a)}}

Se desea entonces, a partir de las funciones de transferencia dadas, determinar
una función de transferencia para $\psi (s)$ y $v_{i} (s)$. Es decir, buscamos
\begin{equation}
  \frac{\psi (s)}{v_{i} (s)}
  \label{eq:fdt-1}
\end{equation}

Observamos entonces que, como se tienen las F. de T. de $\omega (s) / v_{i} (s)$
y de $\psi (s) / \omega (s)$, entonces
\begin{equation}
  \frac{\omega (s)}{v_{i} (s)} \cdot \frac{\psi (s)}{\omega (s)} = \frac{\psi (s)}{v_{i} (s)}
\end{equation}

La cual es la F. de T. que deseamos. De este modo, tenemos que la función de
transferencia que relaciona el ángulo $\psi$ respecto a una entrada de voltaje
$v_{i}$ es:

\begin{equation}
  \frac{\psi (s)}{v_{i} (s)} =
    \frac{b_{0\omega}\left(b_{1\psi}s + b_{0\psi}\right)}
    {\left(a_{1\omega}s + a_{0\omega}\right)\left(a_{2\psi}s^{2} + a_{1\psi}s + a_{0\psi}\right)}
  \label{eq:fdt-2}
\end{equation}

Luego, queremos representar este sistema en ecuaciones de estado, siendo las
variables de estado a utilizar
$\mathbf{x} = \left[x_{1}\ x_{2}\ x_{3}]\right]^{T} = \left[\psi (t)\ \dot\psi (t)\ \omega (t)\right]^{T}$.
Para esto, aplicaremos la Transformada de Laplace Inversa a cada una de las
F. de T. dadas, y luego esto nos permitirá determinar un sistema cuya entrada
sea el voltaje aplicado $v_{i} (t)$. Así, para la primera F. de T. que relaciona
$\omega$ con $v_{i}$:

\begin{align}
  \frac{\omega (s)}{v_{i} (s)} &= \frac{b_{0\omega}}{a_{1\omega}s + a_{0\omega}} \\
  \iff \omega (s) \left(a_{1\omega}s + a_{0\omega}\right) &= b_{0\omega}v_{i} (s)
    \stepcounter{equation}\tag{\theequation}\label{eq:fdt-3}
\end{align}

Aplicamos entonces la T.L.I. a \eqref{eq:fdt-3} y obtenemos:
\begin{align}
  a_{1\omega}\dot\omega (t) + a_{0\omega}\omega (t) &= b_{0\omega}v_{i} (t) \\
  \iff \dot\omega (t) &= -\frac{a_{0\omega}}{a_{1\omega}}\omega (t) + \frac{b_{0\omega}}{a_{1\omega}}v_{i} (t)
\end{align}
\FloatBarrier
\newpage
