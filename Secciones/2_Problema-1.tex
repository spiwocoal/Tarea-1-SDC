\section{Pregunta \texttt{a)}}\label{pregunta-a}

Se desea entonces, a partir de las funciones de transferencia dadas, determinar
una función de transferencia para $\rojo{\psi} (s)$ y $\verd{v_{i}} (s)$. Es decir, buscamos
\begin{equation}
  \frac{\rojo{\psi} (s)}{\verd{v_{i}} (s)}
  \label{eq:fdt-1}
\end{equation}

Observamos entonces que, como se tienen las F. de T. de $\sfrac{\mora{\omega}(s)}{\verd{v_{i}} (s)}$
y de $\sfrac{\rojo{\psi}(s)}{\mora{\omega}(s)}$, entonces
\begin{equation}
  \frac{\mora{\omega} (s)}{\verd{v_{i}} (s)} \cdot \frac{\rojo{\psi} (s)}{\mora{\omega} (s)} = \frac{\rojo{\psi} (s)}{\verd{v_{i}} (s)}
\end{equation}

La cual es la F. de T. que deseamos. De este modo, tenemos que la función de
transferencia que relaciona el ángulo $\rojo{\psi}$ respecto a una entrada de voltaje
$\verd{v_{i}}$ es:

\begin{equation}
  \frac{\rojo{\psi} (s)}{\verd{v_{i}} (s)} =
    \frac{\nara{\nara{b_{0\omega}}}\left(\nara{b_{1\psi}}s + \nara{b_{0\psi}}\right)}
    {\left(\nara{a_{1\omega}}s + \nara{a_{0\omega}}\right)\left(\nara{a_{2\psi}}s^{2} + \nara{a_{1\psi}}s + \nara{a_{0\psi}}\right)}
  \label{eq:fdt-2}
\end{equation}

Luego, queremos representar este sistema en ecuaciones de estado, siendo las
variables de estado a utilizar
\begin{equation}
  \mathbf{x} = \begin{bmatrix}x_{1} \\ x_{2} \\ x_{3}\end{bmatrix}
    = \begin{bmatrix}\rojo{\psi} (t) \\ \rojo{\dot\psi} (t) \\ \mora{\omega} (t) \end{bmatrix}
  \label{eq:var-estado}
\end{equation}
Para esto, aplicaremos la Transformada de Laplace Inversa a cada una de las
F. de T. dadas, y luego esto nos permitirá determinar un sistema cuya entrada
sea el voltaje aplicado $\verd{v_{i}} (t)$. Así, para la primera F. de T. que relaciona
$\mora{\omega}$ con $\verd{v_{i}}$:

\begin{align}
  \frac{\mora{\omega} (s)}{\verd{v_{i}} (s)} &= \frac{\nara{b_{0\omega}}}{\nara{a_{1\omega}}s + \nara{a_{0\omega}}} \\
  \mora{\omega} (s) \left(\nara{a_{1\omega}}s + \nara{a_{0\omega}}\right) &= \nara{b_{0\omega}}\verd{v_{i}} (s)
    \stepcounter{equation}\tag{\theequation}\label{eq:fdt-3}
\end{align}

Aplicamos entonces la T.L.I. a \eqref{eq:fdt-3} y obtenemos:
\begin{align}
  \nara{a_{1\omega}}\mora{\dot\omega} (t) + \nara{a_{0\omega}}\mora{\omega} (t) &= \nara{b_{0\omega}}\verd{v_{i}} (t) \\
  \mora{\dot\omega} (t) &= -\frac{\nara{a_{0\omega}}}{\nara{a_{1\omega}}}\mora{\omega} (t) + \frac{\nara{b_{0\omega}}}{\nara{a_{1\omega}}}\verd{v_{i}} (t)
  \stepcounter{equation}\tag{\theequation}\label{eq:estado-omega}
\end{align}

Luego, repetimos el procedimiento para la F. de T. entre $\rojo{\psi} (s)$ y $\mora{\omega} (s)$.
Entonces:
\begin{align}
  \frac{\rojo{\psi} (s)}{\mora{\omega} (s)} &= \frac{\nara{b_{1\psi}}s + \nara{b_{0\psi}}}
    {\nara{a_{2\psi}}s^{2} + \nara{a_{1\psi}}s + \nara{a_{0\psi}}} \\
  \rojo{\psi} (s) \left(\nara{a_{2\psi}}s^{2} + \nara{a_{1\psi}}s + \nara{a_{0\psi}}\right) &=
    \mora{\omega} (s) \left(\nara{b_{1\psi}}s + \nara{b_{0\psi}}\right)
\end{align}

Aplicamos T.L.I. y se tiene:
\begin{align}
  \nara{a_{2\psi}}\rojo{\ddot\psi} (t) + \nara{a_{1\psi}}\rojo{\dot\psi} (t) + \nara{a_{0\psi}}\rojo{\psi} (t) &=
    \nara{b_{1\psi}}\mora{\dot\omega} (t) + \nara{b_{0\psi}}\mora{\omega} (t) \\
  \rojo{\ddot\psi} (t) &= -\frac{\nara{a_{1\psi}}}{\nara{a_{2\psi}}}\rojo{\dot\psi} (t) -
    \frac{\nara{a_{0\psi}}}{\nara{a_{2\psi}}}\rojo{\psi} (t) + \frac{\nara{b_{1\psi}}}{\nara{a_{2\psi}}}\mora{\dot\omega} (t) +
    \frac{\nara{b_{0\psi}}}{\nara{a_{2\psi}}}\mora{\omega} (t)
  \stepcounter{equation}\tag{\theequation}\label{eq:estado-psi}
\end{align}

Luego, reemplazamos \eqref{eq:estado-omega} en \eqref{eq:estado-psi} y entonces:
\begin{align}
  \rojo{\ddot\psi} (t) &= -\frac{\nara{a_{1\psi}}}{\nara{a_{2\psi}}}\rojo{\dot\psi} (t) -
    \frac{\nara{a_{0\psi}}}{\nara{a_{2\psi}}}\rojo{\psi} (t) +
    \frac{\nara{b_{1\psi}}}{\nara{a_{2\psi}}}\left(-\frac{\nara{a_{0\omega}}}{\nara{a_{1\omega}}}\mora{\omega} (t) + \frac{\nara{b_{0\omega}}}{\nara{a_{1\omega}}}\verd{v_{i}} (t)\right) +
    \frac{\nara{b_{0\psi}}}{\nara{a_{2\psi}}}\mora{\omega} (t) \\
  \rojo{\ddot\psi} (t) &= -\frac{\nara{a_{1\psi}}}{\nara{a_{2\psi}}}\rojo{\dot\psi} (t) -
    \frac{\nara{a_{0\psi}}}{\nara{a_{2\psi}}}\rojo{\psi} (t) +
    \left(\frac{\nara{b_{0\psi}}}{\nara{a_{2\psi}}} - \frac{\nara{b_{1\psi}}\nara{a_{0\omega}}}{\nara{a_{2\psi}}\nara{a_{1\omega}}}\right) \mora{\omega} (t) +
    \frac{\nara{b_{1\psi}}\nara{b_{0\omega}}}{\nara{a_{2\psi}}\nara{a_{1\omega}}} \verd{v_{i}} (t)
  \stepcounter{equation}\tag{\theequation}\label{eq:estado-psi-2}
\end{align}

Luego, utilizamos las variables de estado definidas en \eqref{eq:var-estado}
reemplazando en \eqref{eq:estado-omega} y \eqref{eq:estado-psi-2}, y tenemos
entonces:
\begin{equation}\label{eq:matriz_sistema}
  \begin{bmatrix}\dot x_{1}(t) \\ \dot x_{2}(t) \\ \dot x_{3}(t)\end{bmatrix} =
    \begin{bmatrix}
      0 & 1 & 0 \\
      -\frac{\nara{a_{0\psi}}}{\nara{a_{2\psi}}}   & -\frac{\nara{a_{1\psi}}}{\nara{a_{2\psi}}} & \frac{\nara{b_{0\psi}}}{\nara{a_{2\psi}}} - \frac{\nara{b_{1\psi}}\nara{a_{0\omega}}}{\nara{a_{2\psi}}\nara{a_{1\omega}}} \\
      0 & 0 & -\frac{\nara{a_{0\omega}}}{\nara{a_{1\omega}}}
    \end{bmatrix}
    \begin{bmatrix}x_{1}(t) \\ x_{2}(t) \\ x_{3}(t)\end{bmatrix} +
    \begin{bmatrix}
      0 \\
      \frac{\nara{b_{1\psi}}\nara{b_{0\omega}}}{\nara{a_{2\psi}}\nara{a_{1\omega}}} \\
      \frac{\nara{b_{0\omega}}}{\nara{a_{1\omega}}}
    \end{bmatrix}
    \verd{v_{i}} (t)
\end{equation}

Donde $\verd{v_{i}} (t)$ es la entrada del sistema.

\subsection{Comentarios}

Podemos comentar respecto a esta pregunta que obtener la función de transferencia
entre $\rojo{\psi}(s)$ y $\verd{v_{i}}(s)$ fue sencillo, puesto que pudimos
simplemente multiplicar ambas F. de T. dadas. Sin embargo, luego obtener la
representación en ecuaciones de estado del sistema fue más complicado, debido
a la gran cantidad de constantes con las que se trata.

Además, al momento de realizar las simulaciones para la pregunta \texttt{b)}
(página \pageref{pregunta-b}), nos dimos cuenta que, al pasar las ecuaciones
de estado al sistema matricial de \eqref{eq:matriz_sistema}, nos equivocamos
con algunos de los términos, lo cual, al intentar realizar la simulación,
resultaba en una respuesta totalmente errónea.

\FloatBarrier
\newpage
