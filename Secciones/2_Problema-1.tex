\section{Pregunta \texttt{a)}}

Se desea entonces, a partir de las funciones de transferencia dadas, determinar
una función de transferencia para $\psi (s)$ y $v_{i} (s)$. Es decir, buscamos
\begin{equation}
  \frac{\psi (s)}{v_{i} (s)}
  \label{eq:fdt-1}
\end{equation}

Observamos entonces que, como se tienen las F. de T. de $\omega (s) / v_{i} (s)$
y de $\psi (s) / \omega (s)$, entonces
\begin{equation}
  \frac{\omega (s)}{v_{i} (s)} \cdot \frac{\psi (s)}{\omega (s)} = \frac{\psi (s)}{v_{i} (s)}
\end{equation}

La cual es la F. de T. que deseamos. De este modo, tenemos que la función de
transferencia que relaciona el ángulo $\psi$ respecto a una entrada de voltaje
$v_{i}$ es:

\begin{equation}
  \frac{\psi (s)}{v_{i} (s)} =
    \frac{b_{0\omega}\left(b_{1\psi}s + b_{0\psi}\right)}
    {\left(a_{1\omega}s + a_{0\omega}\right)\left(a_{2\psi}s^{2} + a_{1\psi}s + a_{0\psi}\right)}
  \label{eq:fdt-2}
\end{equation}

Luego, queremos representar este sistema en ecuaciones de estado, siendo las
variables de estado a utilizar
\begin{equation}
  \mathbf{x} = \begin{bmatrix}x_{1} \\ x_{2} \\ x_{3}\end{bmatrix}
    = \begin{bmatrix}\psi (t) \\ \dot\psi (t) \\ \omega (t) \end{bmatrix}
  \label{eq:var-estado}
\end{equation}
Para esto, aplicaremos la Transformada de Laplace Inversa a cada una de las
F. de T. dadas, y luego esto nos permitirá determinar un sistema cuya entrada
sea el voltaje aplicado $v_{i} (t)$. Así, para la primera F. de T. que relaciona
$\omega$ con $v_{i}$:

\begin{align}
  \frac{\omega (s)}{v_{i} (s)} &= \frac{b_{0\omega}}{a_{1\omega}s + a_{0\omega}} \\
  \omega (s) \left(a_{1\omega}s + a_{0\omega}\right) &= b_{0\omega}v_{i} (s)
    \stepcounter{equation}\tag{\theequation}\label{eq:fdt-3}
\end{align}

Aplicamos entonces la T.L.I. a \eqref{eq:fdt-3} y obtenemos:
\begin{align}
  a_{1\omega}\dot\omega (t) + a_{0\omega}\omega (t) &= b_{0\omega}v_{i} (t) \\
  \dot\omega (t) &= -\frac{a_{0\omega}}{a_{1\omega}}\omega (t) + \frac{b_{0\omega}}{a_{1\omega}}v_{i} (t)
  \stepcounter{equation}\tag{\theequation}\label{eq:estado-omega}
\end{align}

Luego, repetimos el procedimiento para la F. de T. entre $\psi (s)$ y $\omega (s)$.
Entonces:
\begin{align}
  \frac{\psi (s)}{\omega (s)} &= \frac{b_{1\psi}s + b_{0\psi}}
    {a_{2\psi}s^{2} + a_{1\psi}s + a_{0\psi}} \\
  \psi (s) \left(a_{2\psi}s^{2} + a_{1\psi}s + a_{0\psi}\right) &=
    \omega (s) \left(b_{1\psi}s + b_{0\psi}\right)
\end{align}

Aplicamos T.L.I. y se tiene:
\begin{align}
  a_{2\psi}\ddot\psi (t) + a_{1\psi}\dot\psi (t) + a_{0\psi}\psi (t) &=
    b_{1\psi}\dot\omega (t) + b_{0\psi}\omega (t) \\
  \ddot\psi (t) &= -\frac{a_{1\psi}}{a_{2\psi}}\dot\psi (t) -
    \frac{a_{0\psi}}{a_{2\psi}}\psi (t) + \frac{b_{1\psi}}{a_{2\psi}}\dot\omega (t) +
    \frac{b_{0\psi}}{a_{2\psi}}\omega (t)
  \stepcounter{equation}\tag{\theequation}\label{eq:estado-psi}
\end{align}

Luego, reemplazamos \eqref{eq:estado-omega} en \eqref{eq:estado-psi} y entonces:
\begin{align}
  \ddot\psi (t) &= -\frac{a_{1\psi}}{a_{2\psi}}\dot\psi (t) -
    \frac{a_{0\psi}}{a_{2\psi}}\psi (t) +
    \frac{b_{1\psi}}{a_{2\psi}}\left(-\frac{a_{0\omega}}{a_{1\omega}}\omega (t) + \frac{b_{0\omega}}{a_{1\omega}}v_{i} (t)\right) +
    \frac{b_{0\psi}}{a_{2\psi}}\omega (t) \\
  \ddot\psi (t) &= -\frac{a_{1\psi}}{a_{2\psi}}\dot\psi (t) -
    \frac{a_{0\psi}}{a_{2\psi}}\psi (t) +
    \left(\frac{b_{0\psi}}{a_{2\psi}} - \frac{b_{1\psi}a_{0\omega}}{a_{2\psi}a_{1\omega}}\right) \omega (t) +
    \frac{b_{1\psi}b_{0\omega}}{a_{2\psi}a_{1\omega}} v_{i} (t)
  \stepcounter{equation}\tag{\theequation}\label{eq:estado-psi-2}
\end{align}

Luego, utilizamos las variables de estado definidas en \eqref{eq:var-estado}
reemplazando en \eqref{eq:estado-omega} y \eqref{eq:estado-psi-2}, y tenemos
entonces:
\begin{equation}
  \begin{bmatrix}\dot x_{1}(t) \\ \dot x_{2}(t) \\ \dot x_{3}(t)\end{bmatrix} =
    \begin{bmatrix}
      0 & 1 & 0 \\
      -\frac{a_{1\psi}}{a_{2\psi}} & \frac{a_{0\psi}}{a_{2\psi}} & \frac{b_{0\psi}}{a_{2\psi}} - \frac{b_{1\psi}a_{0\omega}}{a_{2\psi}a_{1\omega}} \\
      0 & 0 & -\frac{a_{0\omega}}{a_{1\omega}}
    \end{bmatrix}
    \begin{bmatrix}x_{1}(t) \\ x_{2}(t) \\ x_{3}(t)\end{bmatrix} +
    \begin{bmatrix}
      0 \\
      \frac{b_{1\psi}b_{0\omega}}{a_{2\psi}a_{1\omega}} \\
      \frac{b_{0\omega}}{a_{1\omega}}
    \end{bmatrix}
    v_{i} (t)
\end{equation}

Donde $v_{i} (t)$ es la entrada del sistema.

\FloatBarrier
\newpage
