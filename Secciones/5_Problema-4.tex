\section{Pregunta \texttt{d)}}\label{pregunta-d}

Para este apartado nos piden calcular el $\nara{k_{c}}$ que hace al sistema
marginalmente estable, donde igualmente nos ayudaremos de \textit{MATLAB}, con
el codigo disponible en \autoref{lst:problema-d}, para hacer los calculos
de manera symbolica.

Lo primero que necestiamos es encontrar la función del transferencia de el lazo
cerrado para luego encontrar los polos de esta.

Sabemos que para un sistema realimentado como aquel visto en \hyperref[pregunta-c]{\texttt{c)}},
con F. de T. en el lazo directo $G(s)$, y en el lazo de realimentación $r(s)$,
entonces su función de transferencia será

\begin{equation}
    \frac{G(s)}{1 + G(s)r(s)}
\end{equation}

En nuestro caso, se tiene $G(s) = \nara{k_{c} k_{a}} \dfrac{\rojo{\psi}(s)}{\verd{v_i}(s)}$,
y $r(s) = \nara{k_{st}}$. Por lo tanto, simbólicamente nos queda:
\begin{equation}
    \frac{G(s)}{1 + G(s)r(s)} = \dfrac{\nara{k_ck_a} \frac{\nara{\nara{b_{0\omega}}}\left(\nara{b_{1\psi}}s + \nara{b_{0\psi}}\right)}
    {\left(\nara{a_{1\omega}}s + \nara{a_{0\omega}}\right)\left(\nara{a_{2\psi}}s^{2} + \nara{a_{1\psi}}s + \nara{a_{0\psi}}\right)}}{1+\nara{k_ck_a} \frac{\nara{\nara{b_{0\omega}}}\left(\nara{b_{1\psi}}s + \nara{b_{0\psi}}\right)}
    {\left(\nara{a_{1\omega}}s + \nara{a_{0\omega}}\right)\left(\nara{a_{2\psi}}s^{2} + \nara{a_{1\psi}}s + \nara{a_{0\psi}}\right)}\nara{k_{st}}}
\end{equation}

Reordenadando la expresión utilizando \textit{MATLAB} obtenemos:
\begin{equation}\label{eq:fdt-lc}
    \frac{(\nara{k_a k_a k_{st} b_{1\psi} b_{0\psi}})s + \nara{k_a k_a k_{st} b_{0\psi} b_{0\omega}}}{(\nara{a_{2\psi} a_{1\omega}})s^3 + (\nara{a_{1\psi} a_{1\omega} + a_{2\psi} a_{0\omega}})s^2 + (\nara{a_{0\psi} a_{1\omega} + a_{1\psi} a_{0\omega} + k_a k_c k_{st} b_{1\psi} b_{0\omega}})s + \nara{a_{0\psi} a_{0\omega} + K_a k_c k_{st} b_{0\psi} b_{0\omega}}}
\end{equation}

Nos enfocamos entonces en el denominador que es donde se ecuentran lo polos y
utilizamos el criterio de Routh-Hurwitz para analizar la estabilidad. Trabajaremos
entonces con el denominador de \eqref{eq:fdt-lc}, donde $\nara{a_3},\nara{a_2},\nara{a_1},\nara{a_0}$
representan los coeficientes del polinomio:
\begin{align}
    P(s) = \nara{a_3} s^3 + \nara{a_2} s^2 + \nara{a_1} s + \nara{a_0} &
\end{align}

Construimos entonces la tabla para el criterio de Routh-Hurwitz, y observamos
que no hay ceros presentes en la columan pivote:
\begin{equation}
	\begin{array}{c|cc}
		s^3 & \nara{a_3} & \nara{a_1} \\
		s^2 & \nara{a_2} & \nara{a_0} \\
		s^1 & \nara{b_1} & 0 \\
		s^0 & \nara{b_0} & 
	\end{array}
\end{equation}

Donde sabemos que:
\begin{align}
	\nara{b_1} = \frac{\nara{a_2 a_1} - \nara{a_3 a_0}}{\nara{a_2}},\quad
	\nara{b_0} = \nara{a_0}
\end{align}

Para tener estabilidad en el sistema, los coeficientes de la primera columna
deben ser todos positivos. Entonces, la condición $\nara{b_1} > 0$ garantiza
que el sistema vaya a ser estable.

Reemplazamos los coeficientes y obtenemos
\begin{equation}
    \nara{b_1}=\frac{
        \left(\nara{a_{1\psi} a_{1\omega}} + \nara{a_{2\psi} a_{0\omega}}\right)
        \left(\nara{a_{0\psi} a_{1\omega}} + \nara{a_{1\psi} a_{0\omega}} + \nara{k_a k_c k_{st} b_{1\psi} b_{0\omega}}\right) 
        - \nara{a_{2\psi} a_{1\omega}} 
        \left(\nara{a_{0\psi} a_{0\omega}} + \nara{k_a k_c k_{st} b_{0\psi} b_{0\omega}}\right)
    }{
        \nara{a_{1\psi} a_{1\omega}} + \nara{a_{2\psi} a_{0\omega}}
    } > 0
\end{equation}

Resolvemos entonces para $\nara{k_{c}}$, y vemos que para que el sistema sea
estable se necesita que
\begin{align}
	     \nara{k_{c}} &> \dfrac{-(\nara{a_{1\psi} a_{1\omega}}(\nara{a_{0\psi} a_{1\omega}} + \nara{a_{1\psi} a_{0\omega}}) + \nara{ a_{2\psi} a_{0\omega} }(\nara{a_{0\psi} a_{1\omega}} + \nara{a_{1\psi} a_{0\omega}}) - \nara{a_{2\psi} a_{1\omega} a_{0\psi} a_{0\omega}})}  {(\nara{a_{1\psi} a_{1\omega} k_a k_{st} b_{1\psi} b_{0\omega}}) + (\nara{a_{2\psi} a_{0\omega} k_a k_{st} b_{1\psi} b_{0\omega}}) - (\nara{a_{2\psi} a_{1\omega} k_a k_{st} b_{0\psi} b_{0\omega}})} \\
	\iff \nara{k_{c}} &> 0.0222
\end{align}

Para que el sistema sea estable entonces se debe cumplir lo anterior. Pero
como solo nos interesa que sea marginalmente estable, entonces el $\nara{k_{c}}$
que hace que el sistema lo sea es:
\begin{equation}
    \boxed{\nara{k_c} = 0.0222}
\end{equation}
