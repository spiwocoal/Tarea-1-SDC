\section{Pregunta \texttt{e)}}\label{pregunta-e}

Luego, se quiere repetir \hyperref[pregunta-c]{\texttt{c)}}, pero con un controlador
integrador $\sfrac{\nara{k_{c}}}{s}$. Entonces, el lazo de control que se utilizará
será muy similar al de la \autoref{fig:control-proporcional}, pero el bloque 
controlador será diferente. Éste lazo de control se puede observar en la
\autoref{fig:control-integral}.

\begin{figure}[h]
  \centering
  \begin{tikzpicture}
    % Suma
    \node[draw,
        circle,
        minimum size=0.6cm,
        fill=white!50
    ] (sum) at (0,0){};
     
    \draw (sum.north east) -- (sum.south west)
        (sum.north west) -- (sum.south east);
     
    \draw (sum.north east) -- (sum.south west)
    (sum.north west) -- (sum.south east);
     
    \node[left=-2pt] at (sum.center){\tiny $+$};
    \node[below=-1pt] at (sum.center){\tiny $-$};
     
    % Controlador
    \node [block,
        fill=white,
        right=1cm of sum
      ]  (controller) {$\sfrac{\nara{k_{c}}}{s}$};
    
    % Actuador
    \node [block,
        fill=white, 
        right=1.5cm of controller
      ] (actuator) {$\nara{k_{a}}$};
     
    % Sistema H(s)
    \node [block,
        fill=white, 
        right=1.5cm of actuator
      ] (system) {$\sfrac{\rojo{\psi}(s)}{\verd{v_{i}}(s)}$};
     
    % Sensor/Transmisor
    \node [block,
        fill=white, 
        below right= 1cm and -0.25cm of controller
      ]  (sensor) {$\nara{k_{st}}$};
     
    % Arrows with text label
    \draw[-stealth] (sum.east) -- (controller.west)
        node[midway,above]{};
     
    \draw[-stealth] (controller.east) -- (actuator.west) 
        node[midway,above]{$\azul{w}$};
    
    \draw[-stealth] (actuator.east) -- (system.west) 
        node[midway,above]{$\verd{v_{i}}$};
     
    \draw[-stealth] (system.east) -- ++ (1.25,0) 
        node[midway](output){}node[midway,above]{$\rojo{\psi}$};
     
    \draw[-stealth] (output.center) |- (sensor.east);
     
    \draw[-stealth] (sensor.west) -| (sum.south) 
        node[near end,left]{$\rojo{\psi_{m}}$};
     
    \draw (sum.west) -- ++(-1,0) 
        node[midway,above]{$\rojo{\psi_{d}}$};
  \end{tikzpicture}
  \caption{Lazo cerrado de control para el sistema, con controlador integrador.}\label{fig:control-integral}
\end{figure}


Ahora bien,  para encontrar el $\nara{k_c}$ donde se haga marginalmente estable se procedera como lo visto en \hyperref[pregunta-c]{\texttt{c)}} encontrando la función de transferencia del sistema de la \autoref{fig:control-integral} donde se procedera utilizando el calculo simbolico de matlab. 

\begin{equation}
  \frac{G(s)}{1 + G(s)r(s)} = \dfrac{\frac{\nara{k_c}}{s}\nara{k_a} \frac{\nara{\nara{b_{0\omega}}}\left(\nara{b_{1\psi}}s + \nara{b_{0\psi}}\right)}
  {\left(\nara{a_{1\omega}}s + \nara{a_{0\omega}}\right)\left(\nara{a_{2\psi}}s^{2} + \nara{a_{1\psi}}s + \nara{a_{0\psi}}\right)}}
  {1+\frac{\nara{k_c}}{s} \nara{k_a} \frac{\nara{\nara{b_{0\omega}}}\left(\nara{b_{1\psi}}s + \nara{b_{0\psi}}\right)}
  {\left(\nara{a_{1\omega}}s + \nara{a_{0\omega}}\right)\left(\nara{a_{2\psi}}s^{2} + \nara{a_{1\psi}}s + \nara{a_{0\psi}}\right)}\nara{k_{st}}}
\end{equation}

Ahora reordenamos se obtiene que la F.de T. seria la siguiente:


\begin{equation}
  \dfrac{(\nara{b_{1\psi} b_{0\omega} k_{st} k_a k_c})s + \nara{b_{0\psi} b_{0\omega}k_st k_a k_c}}
  {(\nara{a_{2\psi} a_{1\omega}}) s^4 +
  (\nara{a_{1\psi} a_{1\omega}} + \nara{a_{2\psi} a_{0\omega}}) s^3 + 
  (\nara{a_{0\psi} a_{1\omega}} + \nara{a_{1\psi} a_{0\omega}}) s^2 + 
  (\nara{a_{0\psi} a_{0\omega}} + \nara{b_{1\psi} b_{0\omega}k_{st} k_a k_c})s+ 
  \nara{b_{0\psi} b_{0\omega} k_{st} k_a k_c}}
\end{equation}

Donde nos enfocamos en el denominador que es donde se ecuentran lo polos, para analizar la estabilidad usamos el criterio de Routh-Hurwitz, cumpliendo que si no hay ceros en la columna pivote. Trabajaremos el denominador como el polinomio característico:
\begin{align}
    P(s) = \nara{a_4} s^4 + \nara{a_3} s^3 + \nara{a_2} s^2 + \nara{a_1} s + \nara{a_0} &
\end{align}

Donde para aplicar el criterio de Routh-Hurwitz, construimos la siguiente tabla de Routh:
\[
\begin{array}{c|ccc}
s^4 & \nara{a_4} & \nara{a_2} & \nara{a_0} \\
s^3 & \nara{a_3} & \nara{a_1} & 0\\
s^2 & \nara{b_1} & \nara{a_0} & 0\\
s^1 & \nara{c_1} & 0 \\
s^0 & \nara{a_0} & 
\end{array}
\]



Donde los términos \(\nara{b_1}\) y \(\nara{c_1}\) se calculan de la siguiente manera:

\[
\nara{b_1} = \frac{\nara{a_3 a_2} - \nara{a_4 a_1}}{\nara{a_3}}, \quad \nara{c_1} = \frac{\nara{b_1 a_1} - \nara{a_3 a_0}}{\nara{b_1}}
\]

La última fila es simplemente \(\nara{a_0}\).

Para que el sistema sea estable, los coeficientes de la primera columna de la tabla de Routh deben ser todos positivos, es decir, se deben cumplir las siguientes condiciones:

\[
\nara{a_4} > 0, \quad \nara{a_3} > 0, \quad \nara{b_1} > 0, \quad \nara{c_1} > 0, \quad \nara{a_0} > 0
\]


Donde ahora remplazando el polinomio por el denominador de la F. de T. tendriamos que:


\begin{align*}
  \nara{a_4} &= \nara{a_{2\psi} a_{1\omega}} \\
  \nara{a_3} &= \nara{a_{1\psi} a_{1\omega}} + \nara{a_{2\psi} a_{0\omega}} \\
  \nara{a_2} &= \nara{a_{0\psi} a_{1\omega}} + \nara{a_{1\psi} a_{0\omega}} \\
  \nara{a_1} &= \nara{a_{0\psi} a_{0\omega}} + \nara{b_{1\psi} b_{0\omega} k_{st} k_a k_c} \\
  \nara{a_0} &= \nara{b_{0\psi} b_{0\omega} k_{st} k_a k_c} \\
  \nara{b_1} &= \frac{\nara{a_3 a_2} - \nara{a_4 a_1}}{\nara{a_3}} \\
  \nara{c_1} &= \frac{\nara{b_1 a_1} - \nara{a_3 a_0}}{\nara{b_1}}
  \end{align*}

Ahora bien con lo anterior, para que sea marginalmente estable necesitamos que se cumpla  \(\nara{b_1} > 0\)     
Luego trabajadno con $\nara{b_1}$ notamos que solo $k_c$ se encuentra en $a_1$
asi que para realizar el calculo mas facil despejamos en base a eso tal que obtendriamos que: 


\begin{align} 
  \frac{\nara{a_3 a_2} - \nara{a_4 a_1}}{\nara{a_3}} &> 0 \\
  \nara{a_3 a_2} - \nara{a_4 a_1} &> 0 \\
  - \nara{a_4 a_1} &> - \nara{a_3 a_2} \\
  \nara{a_1} &< \frac{\nara{a_3 a_2}}{\nara{a_4}} \\
  \nara{a_{0\psi} a_{0\omega}} + \nara{b_{1\psi} b_{0\omega} k_{st} k_a k_c} &< \frac{\nara{a_3 a_2}}{\nara{a_4}} \\
  \nara{b_{1\psi} b_{0\omega} k_{st} k_a k_c} &< \frac{\nara{a_3 a_2}}{\nara{a_4}} - \nara{a_{0\psi} a_{0\omega}} \\
  \nara{k_c}  &< \frac{\nara{a_3 a_2}}{\nara{a_4 b_{1\psi} b_{0\omega} k_{st} k_a}}  - \frac{\nara{a_{0\psi} a_{0\omega}}}{\nara{b_{1\psi} b_{0\omega} k_{st} k_a}}  \\
\end{align}

Ahora bien si damos el uso del los calculos symbolicos y numericos de \textit{Matlab} (\autoref{lst:problema-e_kc})obtendriamos que: 

\begin{equation}
  \nara{k_c} < 0.0362
\end{equation}

Por lo que se debe cumplir lo anterior para que sea estable, entonces si queremos que nuestro sistema sea marginalmente estable se debe cumplir que:

\begin{equation}
  \boxed{\nara{k_c} = 0.0362}
\end{equation}



























