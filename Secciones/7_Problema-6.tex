\section{Pregunta \texttt{f)}}\label{pregunta-f}

Para poder determinar el equivalente discreto nos aprovechamos de la matriz resultante en \hyperref[pregunta-b]{\texttt{b)}} que son la que se ven a continuación:

\begin{align}
    \pmb{A} &= \begin{bmatrix}
        0 & 1 & 0 \\
        -\frac{\nara{a_{0\psi}}}{\nara{a_{2\psi}}}   & -\frac{\nara{a_{1\psi}}}{\nara{a_{2\psi}}} & \frac{\nara{b_{0\psi}}}{\nara{a_{2\psi}}} - \frac{\nara{b_{1\psi}}\nara{a_{0\omega}}}{\nara{a_{2\psi}}\nara{a_{1\omega}}} \\
        0 & 0 & -\frac{\nara{a_{0\omega}}}{\nara{a_{1\omega}}}
    \end{bmatrix} &
    \pmb{B} &= \begin{bmatrix}
        0 \\
        \frac{\nara{b_{1\psi}}\nara{b_{0\omega}}}{\nara{a_{2\psi}}\nara{a_{1\omega}}} \\
        \frac{\nara{b_{0\omega}}}{\nara{a_{1\omega}}}
    \end{bmatrix}
  \end{align}
  \begin{align}
    \pmb{C} = \begin{bmatrix}
      1 & 0 & 0
    \end{bmatrix}
  \end{align}

Ahora bien por lo que vimos en Sistemas lineales Dinamicos sabemos que los equivalentes discretos de las matrices se puede calcular como:

\[
\mathbf{A_d}= e^{\mathbf{A}T} ,\quad \mathbf{B_d}= \int_{0}^{T} e^{\mathbf{A}(T-\sigma)}\mathbf{B}  \,d\sigma 
\]

Donde apoyandonos en los calculos numericos de \textit{Matlab} obtenemos que las matrices discretas serian las siguientes :


\begin{align}
    \mathbf{A_d} &= \begin{bmatrix}
        -0.1106 & 0.0970 & 0.7400 \\
        -7.6484 &  -0.2385 & 4.5822 \\
        0 & 0 & 0.7386
    \end{bmatrix} &
    \mathbf{B_d} &= \begin{bmatrix}
        0.0090 \\
        0.0710 \\
        0.0046
    \end{bmatrix}
  \end{align}
  \begin{align}
    \pmb{C_{d1}} = \begin{bmatrix}
      1 & 0 & 0
    \end{bmatrix}&
    \pmb{C_{d2}} = \begin{bmatrix}
      0 & 1 & 0
    \end{bmatrix}
\end{align}

Siendo $\pmb{C_{d1}} $, $\pmb{C_{d2}} $ las salidas para conseguir $\rojo{\psi}(kT)$ y $\mora{\omega}(kT)$ respectivamente.

Luego los valores propios del la matriz $A_d$ se calcularon usando el comando de \texttt{aig} de \textit{Matlab} dando los siguientes resultados.
\begin{align}
  valor propio_{0} &=  0.7386 \\
  valor propio_{1,2} &=  -0.1745 \pm 0.8588j
\end{align}

A continuación, con las matrices discretas obtenidas graficaremos en un tiempo de $0 \leq kT \leq 10\ \unit{s}$, con respecto a la entrada dada en \hyperref[pregunta-b]{\texttt{b)}}, pero discretizada quedando como:

\begin{equation}
  \verd{v_{i}}(kT) = \frac{\verd{v_{i0}}}{3}\left(r(kT-1) - r(kT-4)\right)
\end{equation}

A continuacion se no pidieron graficar la entrada $ \verd{v_{i}}(kT) $ y las salidas $ \rojo{\psi}(kT)(grados) $ y $ \mora{\omega}(kT)(rpm) $ resultando lo siguiente:







