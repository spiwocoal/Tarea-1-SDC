\section{Pregunta \texttt{f)}}\label{pregunta-f}

Para poder determinar el equivalente discreto nos aprovechamos de la matriz resultante en \hyperref[pregunta-b]{\texttt{b)}} que son la que se ven a continuación:

\begin{align}
    \pmb{A} &= \begin{bmatrix}
        0 & 1 & 0 \\
        -\frac{\nara{a_{0\psi}}}{\nara{a_{2\psi}}}   & -\frac{\nara{a_{1\psi}}}{\nara{a_{2\psi}}} & \frac{\nara{b_{0\psi}}}{\nara{a_{2\psi}}} - \frac{\nara{b_{1\psi}}\nara{a_{0\omega}}}{\nara{a_{2\psi}}\nara{a_{1\omega}}} \\
        0 & 0 & -\frac{\nara{a_{0\omega}}}{\nara{a_{1\omega}}}
    \end{bmatrix} &
    \pmb{B} &= \begin{bmatrix}
        0 \\
        \frac{\nara{b_{1\psi}}\nara{b_{0\omega}}}{\nara{a_{2\psi}}\nara{a_{1\omega}}} \\
        \frac{\nara{b_{0\omega}}}{\nara{a_{1\omega}}}
    \end{bmatrix}
  \end{align}
  \begin{align}
    \pmb{C} = \begin{bmatrix}
      1 & 0 & 0
    \end{bmatrix}
  \end{align}

Ahora bien por lo que vimos en Sistemas lineales Dinamicos sabemos que los equivalentes discretos de las matrices se puede calcular como:

\[
\mathbf{A_d}= e^{\mathbf{A}T} ,\quad \mathbf{B_d}= \int_{0}^{T} e^{\mathbf{A}(T-\sigma)}\mathbf{B}  \,d\sigma 
\]

Donde apoyandonos en los calculos numericos de \textit{Matlab} obtenemos que nuestras matrices discretas serian las siguientes:


\begin{align}
    \mathbf{A_d} &= \begin{bmatrix}
        -0.1106 & 0.0970 & 0.7400 \\
        -7.6484 &  -0.2385 & 4.5822 \\
        0 & 0 & 0.7386
    \end{bmatrix} &
    \mathbf{B_d} &= \begin{bmatrix}
        0.0090 \\
        0.0710 \\
        0.0046
    \end{bmatrix}
  \end{align}
  \begin{align}
    \pmb{C} = \begin{bmatrix}
      1 & 0 & 0
    \end{bmatrix}
\end{align}

