\section{Pregunta \texttt{h)}}\label{pregunta-h}

%PONDRE LAS COSAS LUEGO ESCRIBO

En este apartado tenemos que determinar el $\nara{k_c}$  que hace el sistema
marginalmente estable y simular como en \hyperref[pregunta-g]{\texttt{g)}}. Para
esto, primero calcularemos la función de tansferencia de el lazo cerrado, donde
primero calculamos la funcion de transferencia de las matrices discretas obtenidas
anteriomente, tal que,
\begin{equation}
   \frac{\rojo{\psi}(z)}{\verd{v_{i}}(z)} = \frac{0.008997z^2 + 0.005767 z - 0.003827}
    {z^3 - 0.3895 z^2 + 0.5102 z -0.5673}
\end{equation}

Ahora bien, para un sistema realimentado se tiene en el lazo directo $G(z)$, y 
en el lazo de realimentación $r(z)$, entonces su función de transferencia será
\begin{equation}
    \frac{G(z)}{1 + G(z)r(z)}
\end{equation}

En nuestro caso, se tiene $G(z) = \nara{k_{c} k_{a}} \frac{\rojo{\psi}(z)}{\verd{v_i}(z)}$,
y $r(z) = \nara{k_{st}}$. Por lo tanto,
\begin{equation}
    \frac{G(z)}{1 + G(z)r(z)} = \dfrac{\nara{k_ck_a} \frac{0.008997z^2 + 0.005767 z - 0.003827}
    {z^3 - 0.3895 z^2 + 0.5102 z -0.5673}}
    {1+\nara{k_ck_a} \frac{0.008997z^2 + 0.005767 z - 0.003827}
    {z^3 - 0.3895 z^2 + 0.5102 z -0.5673}\nara{k_{st}}}
\end{equation}

Reordenadando la expresión utilizando \textit{MATLAB} y obtenemos:
\begin{equation}\label{eq:fdt-LCd}
    \frac{ \nara{k_c k_a} \left( 0.009 z^2 + 0.0058 z - 0.0038 \right)}
    { z^3 + \left( \nara{k_c k_a k_{st}} \cdot 0.009 - 0.3895 \right) z^2 
    + \left( \nara{k_c k_a k_{st}} \cdot 0.0058 + 0.5102 \right) z 
    - \nara{k_c k_a k_{st}} \cdot 0.0038 - 0.5673 }
\end{equation}

Nos enfocamos entonces en el denominador que es donde se ecuentran lo polos y
utilizamos el criterio de Routh-Hurwitz para analizar la estabilidad. 
Trabajaremos entonces con el denominador de \eqref{eq:fdt-LCd}, donde $\nara{a_2},\nara{a_1},\nara{a_0}$
representan los coeficientes del polinomio:
\begin{align}
    P(z) =  z^3 + \nara{a_2} z^2 + \nara{a_1} z + \nara{a_0} &
\end{align}

Para estudiar la estabilidad del sistema ocuparemos el método de transformación
bilineal y el criterio para estabilidad Routh-Hurwitz como mencionamos, entonces
primero definimos la transformación bilineal, en la cual se reemplaza $z$, tal que 
\begin{equation}
    z = \frac{w+1}{w-1}
\end{equation}

Ahora sustituimos $z$ en el polinomio característico:
\begin{equation}
    P(z) = (\frac{w+1}{w-1})^3 + \nara{a_2} (\frac{w+1}{w-1})^2 + \nara{a_1} (\frac{w+1}{w-1}) + \nara{a_0}
\end{equation}

Luego, se multiplica por $(w-1)^3$:
\begin{equation}
    P(z) =  ({w+1})^3 + \nara{a_2} ({w+1})^2 (w-1) + \nara{a_1} ({w+1})(w-1)^2 + \nara{a_0}(w-1)^3
\end{equation}

Resolvemos los productos, distribuimos y reordenamos obteniendo:
\begin{align}
  P(z) &= (\nara{a_0} + \nara{a_1} + \nara{a_2} + 1) w^3 
    + (\nara{a_2} - \nara{a_1} - 3\nara{a_0} + 3) w^2 \nonumber \\ 
    &+ (3\nara{a_0} - \nara{a_1} - \nara{a_2} + 3) w 
    (- \nara{a_0} + \nara{a_1} - \nara{a_2} + 1)
\end{align}

Construimos entonces la tabla para el criterio de Routh-Hurwitz, y observamos
que no hay ceros presentes en la columan pivote:
\begin{equation}
  \begin{array}{c|cc}
    s^3 & (\nara{a_0} + \nara{a_1} + \nara{a_2} + 1) & (3\nara{a_0} - \nara{a_1} - \nara{a_2} + 3)  \\
    s^2 & (\nara{a_2} - \nara{a_1} - 3\nara{a_0} + 3)  & (- \nara{a_0} + \nara{a_1} - \nara{a_2} + 1) \\
    s^1 & \nara{b_1} & 0 \\
    s^0 & \nara{b_0} & 
  \end{array}
\end{equation}

Donde sabemos que:
\begin{align}
  \nara{b_1} = \frac{(\nara{a_2} - \nara{a_1} - 3\nara{a_0} + 3) (3\nara{a_0} - \nara{a_1} - \nara{a_2} + 3) - (\nara{a_0} + \nara{a_1} + \nara{a_2} + 1)(- \nara{a_0} + \nara{a_1} - \nara{a_2} + 1)}{(\nara{a_2} - \nara{a_1} - 3\nara{a_0} + 3)},\quad
  \nara{b_0} = \nara{a_0}
\end{align}

Para tener estabilidad en el sistema, los coeficientes de la primera columna
deben ser todos positivos. Entonces, la condición $\nara{b_1} > 0$ garantiza
que el sistema vaya a ser estable.

Ahora bien reemplazando $\nara{b_1}$ por los valores correspondientes llegamos
a la siguiente inecuación:
\begin{equation}
    \nara{b_1} = \frac{\nara{k_c}^2  +0.0493 \nara{k_c}  -0.000244}{-0.0065 \nara{k_c}  -0.000298} > 0
\end{equation}
-0.0458


A continuación, analizamos las raíces del numerador y del denominador, ya que
el valor de la fracción depende de los signos de ambos. La fracción es positiva
si el numerador y el denominador tienen el mismo signo (ambos positivos o ambos
negativos).
\begin{itemize}
    \item \textbf{Numerador:} \( \nara{k_c}^2  +0.0493 \nara{k_c}  -0.000244 = 0 \)
    \item \textbf{Denominador:} \(-0.0065 \nara{k_c}  -0.000298 = 0 \)
\end{itemize}

Las soluciones para las raíces del numerador y el denominador son las siguientes:
\begin{itemize}
    \item Para el numerador: \( \nara{k_c} = 0.0056 \) y \( \nara{k_c} = 0.0399 \)
    \item Para el denominador: \( \nara{k_c} = 0.0158 \)
\end{itemize}

Ahora, para que la fracción sea mayor que cero, debemos analizar los intervalos
donde el numerador y el denominador tienen el mismo signo.
\begin{enumerate}
    \item \( \nara{k_c} < 0.0056 \)
    \item \( 0.0056 < \nara{k_c} < 0.0158 \)
    \item \( 0.0158 < \nara{k_c} < 0.0399 \)
    \item \( \nara{k_c} > 0.0399 \)
\end{enumerate}

Los signos del numerador y del denominador en los intervalos evaluados son los siguientes:
\begin{itemize}
    \item Para \( \nara{k_c} = 0.005 \): el numerador y el denominador son \textbf{positivos}.
    \item Para \( \nara{k_c} = 0.01 \): el numerador es \textbf{negativo} y el denominador es \textbf{positivo}.
    \item Para \( \nara{k_c} = 0.02 \): tanto el numerador como el denominador son \textbf{negativos}.
    \item Para \( \nara{k_c} = 0.04 \): el numerador es \textbf{positivo} y el denominador es \textbf{negativo}.
\end{itemize}

Por lo tanto, la fracción sera mayor que cero cuando el numerador y el denominador,
tengan el mismo signo. Esto ocurre en los siguientes intervalos:
% \begin{align}
%   &\nara{k_c} < 0.0056 \\ %\quad \text{(numerador y denominador son positivos)} \\
%   0.0158 < &\nara{k_c} < 0.0399 % \quad \text{(numerador y denominador son negativos)}
% \end{align}
\begin{align}
  \nara{k_c} &< 0.0056 \quad \text{(numerador y denominador son positivos)} \\
  0.0158 < \nara{k_c} &< 0.0399 \quad \text{(numerador y denominador son negativos)}
\end{align}

En conclusión, los valores de $\nara{k_c}$ que cumplen la condición son:
\begin{equation}
  \nara{k_c} < 0.0056 \quad \lor \quad 0.0158 < \nara{k_c} < 0.0399
\end{equation}

%0.0399 no es por que hace negativo la columna pivote por lo que creo que es 0.0056

\subsection{Comentarios}

Aqui van a ir los comentarios \textbf{NO OLVIDAR!!}.
