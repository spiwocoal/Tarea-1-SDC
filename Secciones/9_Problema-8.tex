\section{Pregunta \texttt{h)}}\label{pregunta-h}


En este apartado tenemos que determinar el $\nara{k_c}$  que hace el sistema
marginalmente estable y simular como en \hyperref[pregunta-g]{\texttt{g)}}. Para
esto, primero calcularemos la función de tansferencia de el lazo cerrado, donde
primero calculamos la funcion de transferencia de las matrices discretas obtenidas
anteriomente, tal que,
\begin{equation}
   \frac{\rojo{\psi}(z)}{\verd{v_{i}}(z)} = \frac{0.008997z^2 + 0.005767 z - 0.003827}
    {z^3 - 0.3895 z^2 + 0.5102 z -0.5673}
\end{equation}

Ahora bien, para un sistema realimentado se tiene en el lazo directo $G(z)$, y 
en el lazo de realimentación $r(z)$, entonces su función de transferencia será
\begin{equation}
    \frac{G(z)}{1 + G(z)r(z)}
\end{equation}

En nuestro caso, se tiene $G(z) = \nara{k_{c} k_{a}} z^{-1} \cdot \frac{\rojo{\psi}(z)}{\verd{v_i}(z)}$,
y $r(z) = \nara{k_{st}}$. Por lo tanto,
\begin{equation}
    \frac{G(z)}{1 + G(z)r(z)} = \dfrac{\nara{k_ck_a} z^{-1} \cdot\frac{0.008997z^2 + 0.005767 z - 0.003827}
    {z^3 - 0.3895 z^2 + 0.5102 z -0.5673}}
    {1+\nara{k_ck_a} z^{-1} \frac{0.008997z^2 + 0.005767 z - 0.003827}
    {z^3 - 0.3895 z^2 + 0.5102 z -0.5673}\nara{k_{st}}}
\end{equation}

Reordenadando la expresión utilizando \textit{MATLAB} y obtenemos:
\begin{equation}\label{eq:fdt-LCd}
        \frac{ \nara{k_c k_a} \left( 0.009 z^2 + 0.0058 z - 0.0038 \right)}
        { z^4 -0.3895z^3 + \left( \nara{k_c k_a k_{st}} \cdot 0.009 + 0.5102
        \right) z^2 
        + \left( \nara{k_c k_a k_{st}} \cdot0.0058 -0.5673 \right) z 
        - \nara{k_c k_a k_{st}} \cdot 0.0038 }
\end{equation}

Nos enfocamos entonces en el denominador que es donde se ecuentran lo polos y
utilizamos el criterio de Routh-Hurwitz para analizar la estabilidad. 
Trabajaremos entonces con el denominador de \eqref{eq:fdt-LCd}, donde $\nara{a_3},\nara{a_2},\nara{a_1},\nara{a_0}$
representan los coeficientes del polinomio:
\begin{align}
    P(z) = z^4 +\nara{a_3}z^3 + \nara{a_2} z^2 + \nara{a_1} z + \nara{a_0} &
\end{align}

Para estudiar la estabilidad del sistema ocuparemos el método de transformación
bilineal y el criterio para estabilidad Routh-Hurwitz como mencionamos, entonces
primero definimos la transformación bilineal, en la cual se reemplaza $z$, tal que 
\begin{equation}
    z = \frac{w+1}{1-w}
\end{equation}

Ahora sustituimos $z$ en el polinomio característico:
\begin{equation}
    P(z) = (\frac{w+1}{1-w})^4 + \nara{a_3}(\frac{w+1}{1-w})^3 + \nara{a_2} (\frac{w+1}{1-w})^2 + \nara{a_1} (\frac{w+1}{1-w}) + \nara{a_0}
\end{equation}

Luego, se multiplica por $(1-w)^4$:
\begin{equation}
    P(z) = ({w+1})^4+ \nara{a_3}({w+1})^3(1-w)  + \nara{a_2} ({w+1})^2 (1-w) ^2+ \nara{a_1} ({w+1})(1-w)^3 + \nara{a_0}(1-w)^4
\end{equation}


Resolvemos los productos, distribuimos y reordenamos obteniendo:
\begin{align}
  P(z) &= (\nara{a_0} - \nara{a_1} + \nara{a_2} -\nara{a_3} + 1)w^4 +( 2\nara{a_1} - 4\nara{a_0} - 2 \nara{a_3} +4 ) w^3  
    + ( 6\nara{a_0} - 2\nara{a_2} +6 ) w^2 \nonumber \\
    &+ (2\nara{a_3} - 2\nara{a_1} - 4\nara{a_0} + 4 ) w +
    (\nara{a_0} + \nara{a_1} + \nara{a_2} +\nara{a_3} + 1)
\end{align}

Construimos entonces la tabla para el criterio de Routh-Hurwitz, y observamos
que no hay ceros presentes en la columan pivote:
\begin{equation}
  \begin{array}{c|ccc}
    w^4 & (\nara{a_0} - \nara{a_1} + \nara{a_2} -\nara{a_3} + 1) & ( 6\nara{a_0} - 2\nara{a_2} +6 )  &  (\nara{a_0} + \nara{a_1} + \nara{a_2} +\nara{a_3} + 1)\\
    w^3 &( 2\nara{a_1} - 4\nara{a_0} - 2 \nara{a_3} +4 )   &(2\nara{a_3} - 2\nara{a_1} - 4\nara{a_0} + 4 )  & 0\\
    w^2 & \nara{b_1} &  (\nara{a_0} + \nara{a_1} + \nara{a_2} +\nara{a_3} + 1)\\
    w^1 & \nara{b_0} &  \\
    w^0 &    (\nara{a_0} + \nara{a_1} + \nara{a_2} +\nara{a_3} + 1)  & 
  \end{array}
\end{equation}

Donde sabemos que:
\begin{align}
  \nara{b_1} &= \frac{( 2\nara{a_1} - 4\nara{a_0} - 2 \nara{a_3} +4 )( 6\nara{a_0} - 2\nara{a_2} +6 )- (\nara{a_0} - \nara{a_1} + \nara{a_2} -\nara{a_3} + 1)(2\nara{a_3} - 2\nara{a_1} - 4\nara{a_0} + 4 ) }{( 2\nara{a_1} - 4\nara{a_0} - 2 \nara{a_3} +4 ) },\\ 
  \nara{b_0} &= \frac{\nara{b_1} (2\nara{a_3} - 2\nara{a_1} - 4\nara{a_0} + 4 ) -( 2\nara{a_1} - 4\nara{a_0} - 2 \nara{a_3} +4 ) (\nara{a_0} + \nara{a_1} + \nara{a_2} +\nara{a_3} + 1) }{\nara{b_1}}
\end{align}

Para tener estabilidad en el sistema, los coeficientes de la primera columna
deben ser todos positivos. Entonces, la condición $\nara{b_1} > 0 \land \nara{b_0>0}$ garantiza
que el sistema vaya a ser estable.

Ahora bien reemplazando $\nara{b_1}$ por los valores correspondientes llegamos
a la siguiente inecuación: %%
\begin{equation}
    \nara{b_1} = \frac{\nara{k_c}^2  +0.0034 \nara{k_c}  -0.000207}{-0.0043
    \nara{k_c}  -0.0001017} > 0
\end{equation}
\begin{equation}
    \nara{b_0} = \frac{\nara{k_c}^3  + 0.1088\nara{k_c}^2  +0.000827 \nara{k_c} +0.0000112}{0.0161\nara{k_c}^2   +0.0000554\nara{k_c} -000000333} > 0
\end{equation}

Donde con \textit{MATLAB} obtenemos los valores de $\nara{k_c}$ que cumplen la condición por separado siendo los siguientes:

\vspace*{0.2cm}

    \begin{multicols}{2}
        Para $\nara{b_1}$: 
        \begin{align}
            \nara{k_{c0}} & = -0.0162 \\
            \nara{k_{c1}} & = 0.0128
        \end{align}

        \columnbreak
        
        Para $\nara{b_0}$:
        \begin{align}
            \nara{k_{c2}} & = 0.0069 \\
            \nara{k_{c3}} & = -0.0163 \\
            \nara{k_{c4}} & = -0.0163
        \end{align}
    \end{multicols}

Probando los valores donde \( k_c > 0 \) en \textit{MATLAB}, vemos cuál cumple la condición del criterio de Routh-Hurwitz, que consiste en que la columna pivote sea completamente positiva.

\begin{equation}
    \nara{k_{c}}  < 0.0069
\end{equation}

Ahora bien para que el sistema sea estable entonces se debe cumplir lo anterior. Pero 
como solo nos interesa que sea marginalmente estable, entonces el $\nara{k_{c}}$
que hace que el sistema lo sea es:
\begin{equation}
    \boxed{\nara{k_c} = 0.0069}
\end{equation}

Ahora bien para determinar el perido de oscilación primero primero econtramos los valores propios como en  \hyperref[pregunta-c]{\texttt{c)}}.

Entonces, los valores propios del sistema serán los polos de la función de
transferencia. Buscamos entonces
\begin{equation}
  1 + G(z)r(z) = 0
\end{equation}

Con, $G(z) =  \nara{k_{c} k_{a}} z^{-1} \dfrac{\rojo{\psi}(z)}{\verd{v_{i}(z)}}$, y
$r(s) = \nara{k_{st}}$. Entonces, resolvemos la ecuación, y obtenemos que los
autovalores del sistema en L.C. discreto son:
\begin{align*}
  \lambda_{0} &= 0.5978\\
  \lambda_{1} &= -0.2488\\
  \lambda_{2,3} &= 0.0203 \pm 0.9975j
\end{align*}

Entonces, el periodo de oscilación es
\begin{equation}
    T = \frac{2\pi}{\mathfrak{Im}(\lambda_{2})} = 6.298\ \unit{T s} 
\end{equation}

Como $T= 0.2$ es decir cada $1.259 \unit{s}$ 

\subsection{Comentarios}


Con respecto al desarrollo, podemos notar que la función de transferencia (\eqref{eq:fdt-LCd}) es de orden 4. Esto se debe a que el retardo por cálculo ($z^{-1}$) introduce un polo adicional en la función de transferencia del lazo cerrado, incrementando su orden y afectando la dinámica del sistema.

Además, al analizar los gráficos de salida con respecto a la entrada de referencia, observamos que el sistema no logra alcanzar el valor de referencia. Este comportamiento se debe a que estamos utilizando un controlador proporcional $\nara{k_c}$, el cual ajusta la salida del sistema en función de la magnitud del error en tiempo real, pero no es capaz de eliminar completamente el error en estado estacionario. El controlador proporcional genera una corrección proporcional a la diferencia entre la salida y la referencia, pero en sistemas donde se requiere un seguimiento preciso , el error residual es inevitable debido a la ausencia de acción integral. Por lo tanto, el sistema no alcanza la altura deseada debido a este error en estado estacionario.




